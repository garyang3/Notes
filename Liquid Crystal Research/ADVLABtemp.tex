\documentclass[11pt]{article} 

\usepackage{graphicx}
\usepackage{amsmath,amssymb}
\usepackage{tikz}
\usepackage{textcomp}
\usepackage[shortlabels]{enumitem}

\newcommand{\doctitle}{ }
\makeatletter
\renewcommand{\ps@headings}{%
\renewcommand{\@evenhead}{\parbox{\textwidth}{\hrulefill
\fbox{\sffamily LC 2022}\hrulefill } }%
\renewcommand{\@oddhead}{\parbox{\textwidth}{\hrulefill 
\fbox{\sffamily LC 2022}\hrulefill}}}
\makeatother


\linespread{.9}
\hoffset=-0in    \voffset=-.5in
\oddsidemargin=0in   \evensidemargin=0in
\topmargin=-.25in
\textwidth=6.5in   \textheight=9.5in
\columnseprule=.3pt  
\setlength{\parskip}{1em}



\usepackage{multirow}
\usepackage{multicol} 
\usepackage{braket}

\newcommand{\mytitle}[1]{{
\begin{centering}
{\Large \sffamily LC 2022}\\
\bigskip\bigskip{\Large \sffamily \bfseries{#1}}\\
\bigskip\bigskip\end{centering}
}}

\newcommand{\mytitlecompact}[1]{{

\hfill
{\Large \sffamily \bfseries{#1}}
\hfill
%\bigskip
}}


\newcommand{\mysection}[1]{{
\smallskip
\noindent
{\large \sffamily \bfseries{#1}}
}}


\newcommand{\mysubsection}[1]{{
\smallskip
{\sffamily \bfseries{#1}}
}}

\pagestyle{empty}
\pagestyle{headings}

\def\C{{\mathbb{C}}}
\def\R{{\mathbb{R}}}
\def\Q{{\mathbb{Q}}}
\def\Z{{\mathbb{Z}}}
\def\N{{\mathbb{N}}}
\def\cA{{\mathcal{A}}}
\def\cC{{\mathcal{C}}}
\newcommand{\expectation}[1]{\langle #1\rangle}
%\def\bra{\langle}
%\def\ket{\rangle}

\begin{document}
%\newcommand⟨\expectation⟩[1][]{$\langle$ #1 $\rangle$}

\mytitlecompact{Liquid Cyrstal Measurements Notes}
%%
\section*{Wedge Cell Measurements}
(Mostly Aronzon Thesis)

\subsection*{$\alpha$}
Put the empty wedge cell on the track, shine laser on it. Put a sheet of paper next to the laser. The distance between the reflected dot and
the laser is $l$. The distance between cell and laser is $L$. Then the angle the top of wedge cell makes is $\alpha$ and satisfies
\begin{equation}
    \tan(2\alpha) = \frac{l}{L}.
\end{equation}
\subsection*{n}
Put a card at the far end of the track, shine laser through the empty wedge cell, mark the dot on the card. Fill the wedge cell with sample, shine laser
through it, the displacement of the dot is, say $\Delta l$. Then we have 
\begin{equation}
    \tan{\delta} = \frac{\Delta l}{L}. 
\end{equation}
And \begin{equation}
    \boxed{n = \frac{\sin{(\frac{\alpha+\delta}{2})}}{\sin{(\frac{\alpha}{2})}}}
\end{equation}
\section*{Index Determination}
\subsection*{Index of Water}
As our target solution are mostly water, we should at first try to find the relationship 
between $n_{water}$ and $n_{sample}$. We can analytically calculate the $n_{water}$ by solving
\begin{equation}
    \frac{n_{water}^2-1}{n_{water}^2+1} (\frac{1}{\bar{\rho}})= a_0+a_1\bar{\rho}+a_2\bar{T}+a_3\bar{\lambda}^2\bar{T}
    +\frac{a_4}{\bar{\lambda}}^2+\frac{a_5}{\bar{\lambda}^2-\bar{\lambda}_{UV}^2}+\frac{a_6}{\bar{\lambda}^2-\bar{\lambda}_{IR}^2}+
a_7 \bar{\rho}^2,
    \label{Index Of Refraction for Water}
\end{equation}
where $\bar{T} = \frac{T}{T^*}$, $\bar{\rho} = \frac{\rho}{\rho^*}$, and $\bar{\lambda} = \frac{\lambda}{\lambda^*}$. (Type in exactly what these quantities are later)

\subsection*{$n_{\text{iso}}, n_{\perp}, n_{\parallel}$}
From the property 
\begin{equation}
    n_{\text{iso}} = \frac{n_{\parallel}+2n_{\perp}}{3}
\end{equation}
and $n_{\parallel} - n_{\perp} = \Delta n$, we may deduce
\begin{equation}
    \boxed{n_{\parallel} = n_{\text{iso}} + \frac{2}{3} \Delta n}
\end{equation}
\begin{equation}
    \boxed{n_{\perp} = n_{\text{iso}} -\frac{1}{3} \Delta n}
\end{equation}

\subsection*{Calculating $n$ at various temperature and wavelength}
We have learned that from data given in Nastishi05, typically, ($n_{\text{iso}}-n_{\text{water}}\approx 0.03$).
Therefore, what is left for us to do is to figure out how $\Delta n$ various with temperature and wavelength. The wavelength dependence is given in
Nastishi05, for $15\%$ and $17\%$. What we did is just interpolated a line for $16\%$. This gives $\Delta n$ for various wavelength at $20^\circ C$. 
The temperature dependece is generally given by the rule 
\begin{equation}
    \Delta n = \Delta n_0 (1-\frac{T}{T^*})^{\gamma}
\end{equation}
where $\Delta n_0, T^{*}$, and $\gamma$ are fit parameters. We acquired a fit for the function, and get 
\[\Delta n_0 = -0.022\], $T^* = 31.75^\circ C$, and $\gamma = 0.082$. 
To account for the difference in transition temperature between our sample
and used in the paper, we modified the equation to be (for $\lambda=633$ nm):
\begin{equation}
    \Delta n = \Delta n_{20^\circ C} (1-\frac{T-1.3}{T^*})^{\gamma} \label{delta n 633, with t}
\end{equation}
to account for difference between transition temperature and plugged in exact value for $\Delta n_0 = Delta n_{20^\circ C}$ found in Nastishin05.

Then, with the assumption that 
\begin{equation}
    \frac{\Delta n_{\lambda\text{, } 20^\circ\text{C}}}{\Delta n_{\lambda\text{, } t}}=
    \frac{\Delta n_{633\text{nm, } 20^\circ\text{C}}}{\Delta n_{633\text{nm, } t}} \label{generalize delta n for different wavelength to temperature}
\end{equation}
We can use \ref{delta n 633, with t} to find $Delta n_{633\text{nm, } t}$, and with these, we are able to find $n_\perp$ and $n_\parallel$ for all wavelength and all temperatures. 

\section*{Measurement Protocal}
The detailed description of how to use the microscope. 
\begin{enumerate}
    \item No sample, set on parallel polarizers. 
    
    Post Processing $\leftarrow$ Shading Correction $\leftarrow$ Define. 


    \item Add disred filter, parallel polarizer. Add water sample. Adjust exposure time and power to make desired curve fill up the spectrum. 
    
    \item Measure $I_\perp^w$ and $I_\parallel^w$. Also put a blocker under the microscope to measure $I_\perp^0$ and $I_\parallel^w0.$
    \item Switch to sample. Set sample to desired temperature. Wait 2 minutes after temperature control hovers the desired limit. Be sure to selected the 
the region that is darkest with crossed polarizer. Measure $I_\perp^s$ and $I_\parallel^s$ and $I_\perp^0$ and $I_\parallel^0$. 
    \item After traversing all temperature, switch back to water and do the same measurement to make sure isotropicity is not violated. 
    \item At high temperature (isotropic phase for samples), make measurement for sample again. 
\end{enumerate}
Empirically, a longer exposure time helps with averaging intensity. 
The absorption at each temperature is 
\begin{equation}
    A_\parallel = \log_{10} \frac{I_\parallel^w-I_\parallel^0}{I_\parallel^s-I_\parallel^0},
\end{equation}
and 
\begin{equation}
    A_\perp = \log_{10} \frac{I_\perp^w-I_\perp^0}{I_\perp^s-I_\perp^0},
\end{equation}
\section*{Calculate $S$}
Finally, we may calculate the order parameter
\begin{equation}
    \boxed{S = \frac{n_{\parallel}A_{\parallel}-n_{\perp}A_{\perp}}{n_{\parallel}A_{\parallel}+2n_{\perp}A_{\perp}}}
\end{equation}


\end{document}
 



